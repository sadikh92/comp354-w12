\documentclass[12pt]{article}
\usepackage{graphicx}
\usepackage{multirow}

\pagestyle{empty}
\setcounter{secnumdepth}{2}

\topmargin=0cm
\oddsidemargin=0cm
\textheight=22.0cm
\textwidth=16cm
\parindent=0cm
\parskip=0.15cm
\topskip=0truecm
\raggedbottom
\abovedisplayskip=3mm
\belowdisplayskip=3mm
\abovedisplayshortskip=0mm
\belowdisplayshortskip=2mm
\normalbaselineskip=12pt
\normalbaselines

\begin{document}

\vspace*{0.5in}
\centerline{\bf\Large Design Document}

\vspace*{0.5in}
\centerline{\bf\Large Team 3}

\vspace*{0.5in}
\centerline{\bf\Large 11 March 2012}

\vspace*{1.5in}
\begin{table}[htbp]
\caption{Team}
\begin{center}
\begin{tabular}{|r | c|}
\hline
Name & ID Number \\
\hline\hline
John Alexander Landovskis & 9641394\\
Eric Regnier & 9332219\\
Rahmat Jaffari & 9095926\\
Bacacar Ndiaye & 9744258\\
Kam Shing Yip & 9602518\\
David Campbell & Y\\
Adam Anderson & Y\\
Chris Gee & Y\\
Eric Dana & Y\\
\hline
\end{tabular}
\end{center}
\end{table}

\clearpage

\section{Introduction}

The introduction of the document provides an overview of the entire document,
briefly introducing what are its goals, and what information is to be found in it.


\section{Architectural Design} \label{sec:arch}



\subsection{Architectural Diagram}
\vspace*{0.5in}
\includegraphics{diagrams/package_diagram}
\centerline {Figure 1: Package diagram}
\vspace*{0.2in}

Our system is designed with extensibility and scalability in mind.  We are taking great care in designing a framework which can be updated easily.  Many of the anticipated changes to our system in future phases will only require adding new types of data and changing the user presentation code to make use of these new data.  The framework we have designed will only require "plugging in" these new types of data without refactoring the logic that allows the user to view and manipulate the entered data, retrieves and updates the save file, etc.  There are four basic, logical components of the system: the Controller, the Models, the Views, and the Util.

\begin{itemize}

\item {\bf Controller}: The main controller that runs the tardis task manager.

\item {\bf Models}: Reader and Writer methods.

\item {\bf Views}: Display the output in one of four views.

\item {\bf Util} : Logger and input validators

\end{itemize}



\subsection{Subsystem Interface Specifications}

Specification of the software interfaces between the subsystems,
i.e. specific messages (or function calls) that are exchanged by the subsystems.
These are also often called ``Module Interface Specifications''.
Description of the parameters to be passed into these function calls in order to have a service fulfilled,
including valid and invalid ranges of values.
Each subsystem interface must be presented in a separate subsection.


\section{Detailed Design} \label{sec:detail}

Complete description of the system design, describing one subsystem separately in respective subsection.
UML class diagrams are to be used, as well as a short textual description describing the purpose of each class.

\subsection{Subsystem $<$Models$>$}

\subsubsection{Detailed Design Diagram}
\includegraphics{subsystems/diagrams/models_class_diagram}

This package stores all the classes that belong to the modeling layer of the system. Modeling classes are classes that are used to read and write the data from/to the XML files. It contains the Person, Task, IPeopleReader, PeopleReader, ITaskReader, TaskReader, ITaskWriter, TaskWriter, FileHelper and XPathHelper classes.

\subsubsection{Units Description}

\emph{Person}\\
Functions:\\
\begin{tabular}{| l | l | l | l |}
\hline
Name & Parameters & Pre-conditions & Post-conditions\\
\hline
		String getAddress		& None       			& None			& Returns the address\\
\hline
		String getCity			& None				& None       	 	& Returns the city \\
\hline
		String getCountry		& None				& None       	 	& Returns the country \\
\hline
		String getFirstName		& None				& None       	 	& Returns the firstName \\
\hline
		String getLastName		& None				& None       	 	& Returns the lastName \\
\hline
		String getPersonId		& None				& None       	 	& Returns the personId \\
\hline
		String getPhoneNumber	& None				& None       	 	& Returns the phoneNumber \\
\hline
		String getPostalCode		& None				& None       	 	& Returns the postalCode \\
\hline
		String getProvince		& None				& None       	 	& Returns the province \\
\hline
		ArrayList$<$Task$>$ getTasks		& ArrayList$<$Task$>$ tasks	& Requires the list 	& Returns all the tasks \\
					& 				& of all tasks 		& for this person \\
\hline
		void setAddress		& String address  		& None			& Sets the address\\
\hline
		void setCity			& String city			& None       	 	& Sets the city \\
\hline
		void setCountry		& String country		& None       	 	& Sets the country \\
\hline
		void setFirstName		& String firstName		& None       	 	& Sets the firstName \\
\hline
		void setLastName		& String lastName		& None       	 	& Sets the lastName \\
\hline
		void setPersonId		& String personId		& None       	 	& Sets the personId \\
\hline
		void setPhoneNumber	& String phoneNumber	& None       	 	& Sets the phoneNumber\\
\hline
		void setPostalCode		& String postalCode		& None       	 	& Sets the postalCode\\
\hline
		void setProvince		& String province		& None       	 	& Sets the province
\\
\hline
\end{tabular}
\\

Attributes:\\
\begin{tabular}{| l | l |}
\hline
 Name                                       & Description\\
\hline
String address		 	 & Address of this person\\
\hline
String city		 	 & City of this person\\
\hline
String country		 	 & Country of this person\\
\hline
String firstName		  & First name of this person\\
\hline
String lastName		  & Last name of this person\\
\hline
String personId		  & Id of this person\\
\hline
String phoneNumber		  & Phone number of this person\\
\hline
String postalCode		  & Postal code of this person\\
\hline
String province		  & Province of this person\\
\hline
\end{tabular}\\
\\

Purpose: Its purpose is to hold the information that is read from the XML file for a particular person.
\\
\\

\emph{Task}\\
Functions:\\
\begin{tabular}{| l | l | l | l |}
\hline
Name & Parameters & Pre-conditions & Post-conditions\\
\hline
		Task				& None       			& None			& Default constructor, \\
						& 	       			& 			& initializes subtasks \\
\hline
		void addSubtask		& Task task			& None       	 	& Adds subtask to this task \\
\hline
		String getDeliverable		& None				& None       	 	& Returns the deliverable \\
\hline
		String getDueDate		& None				& None       	 	& Returns the dueDate \\
\hline
		String getDuration		& None				& None       	 	& Returns the duration \\
\hline
		String getPersonId		& None				& None       	 	& Returns the person Id \\
\hline
		String getShortDescription	& None				& None       	 	& Returns the short description \\
\hline
		String getSubtasks		& None				& None       	 	& Returns the subtasks list \\
\hline
		String getSuperTask		& None				& None       	 	& Returns the super task \\
\hline
		String getTaskId		& None		  		& None			& Returns the task Id \\
\hline
		String getTitle			& None				& None       	 	& Returns the title \\
\hline
		void setDeliverable		& String deliverable		& None       	 	& Sets the deliverable \\
\hline
		void setDueDate		& String dueDate		& None       	 	& Sets the dueDate \\
\hline
		void setDuration		& String duration		& None       	 	& Sets the duration \\
\hline
		void setPersonId		& String personId		& None       	 	& Sets the personId\\
\hline
		void setShortDescription	& String shortDescription	& None       	 	& Sets the shortDescription\\ 
\hline
		void setSuperTask		& String superTask		& None       	 	& Sets the superTask \\
\hline
		void setTaskId			& String taskId	  		& None			& Sets the taskId \\
\hline
		void setTitle			& String title			& None       	 	& Sets the title
\\
\hline
\end{tabular}
\\

Attributes:\\
\begin{tabular}{| l | l |}
\hline
 Name                                     	 	& Description\\
\hline
String deliverable			& Deliverable of this task\\
\hline
String dueDate			& Due date of this task\\
\hline
String duration		 		& Duration of this task\\
\hline
String personId			& ID of the person this task belongs to\\
\hline
String shortDescription		& Short description of this task\\
\hline
ArrayList$<$Task$>$ subTasks	& Sub-tasks of this task\\
\hline
String superTask			& Super task of this task\\
\hline
\end{tabular}\\
\\

Purpose: Its purpose is to hold the information that is read from the XML file for a particular task and its sub-tasks. If the object has a superTask, then this task is a subtask of that superTask.
\\
\\

\emph{IPeopleReader}\\
Functions:\\
\begin{tabular}{| l | l | l | l |}
\hline
Name & Parameters & Pre-conditions & Post-conditions\\
\hline
		ArrayList$<$Person$>$loadPeople 	& String path       			& Requires the path 	& Returns the list of people\\
                                                                                    &                         			 & to the people XML file	& 
\\
\hline
\end{tabular}
\\

Purpose: The interface for loading the people XML file.
\\
\\
\emph{PeopleReader}\\
Functions:\\
\begin{tabular}{| l | l | l | l |}
\hline
Name & Parameters & Pre-conditions & Post-conditions\\
\hline
		PeopleReader				& IInputValidator 	& None				& Default constructor, \\
							& inputValidator	&  requires inputValidator	& \\
\hline
		ArrayList$<$Person$>$loadPeople 	& String path       	& Requires the path,		& Returns all people\\
                                                                                    &                         	& returns a list of people 	 &\\
\hline
		Person loadPerson 			& JXPathContext path & Requires the context	& Returns a Person \\
                                                                                    & personCtx                & pointer			& 
\\
\hline
\end{tabular}
\\

Attributes:\\
\begin{tabular}{| l | l |}
\hline
 Name                                     	 	& Description\\
\hline
IInputValidator inputValidator		& Input validator for reading the XML file\\
\hline
\end{tabular}\\
\\

Purpose: Its purpose is load all the people from the specified XML file path.
\\
\\

%%%%%%%%%%%%%%%%%%%%%%%%%%%%%%
\emph{ITaskReader}\\
Functions:\\
\begin{tabular}{| l | l | l | l |}
\hline
Name & Parameters & Pre-conditions & Post-conditions\\
\hline
		ArrayList$<$Task$>$loadTasks 	& String path       			& Requires the path 	& Returns the list of task\\
                                                                                    &                         			 & to the tasks XML file	& 
\\
\hline
\end{tabular}
\\

Purpose: The interface for loading the tasks XML file.
\\
\\
\emph{TaskReader}\\
Functions:\\
\begin{tabular}{| l | l | l | l |}
\hline
Name & Parameters & Pre-conditions & Post-conditions\\
\hline
		TaskReader				& IInputValidator 	& None				& Default constructor, \\
							& inputValidator	&  requires inputValidator	& \\
\hline
		ArrayList$<$Task$>$loadTasks 	& String path       	& Requires the path,		& Returns all tasks\\
                                                                                    &                         	& returns a list of taks 	 &\\
\hline
		Task loadTask	 			& JXPathContext path & Requires the context	& Returns a Task \\
                                                                                    & taskCtx                	& pointer			& 
\\
\hline
\end{tabular}
\\

Attributes:\\
\begin{tabular}{| l | l |}
\hline
 Name                                     	 	& Description\\
\hline
IInputValidator inputValidator		& Input validator for reading the XML file\\
\hline
\end{tabular}\\
\\

Purpose: Its purpose is load all the taks from the specified XML file path.
\\
\\

%%%%%%%%%%%%%%%%%%%%%%%%%%%%%%
\emph{ITasksWriter}\\
Functions:\\
\begin{tabular}{| l | l | l | l |}
\hline
Name & Parameters & Pre-conditions & Post-conditions\\
\hline
		boolean writeTasks 	& String path       			& Requires the path 		& Returns a flag to\\
                                                           & 					& to the tasks XML file		& indicate if save \\
\hline
                                                           & ArrayList$<$Task$>$ tasks		& Requires list	of tasks	& was succesful
\\
\hline
\end{tabular}
\\

Purpose: The interface for writing the tasks XML file.
\\
\\
\emph{TasksWriter}\\
Functions:\\
\begin{tabular}{| l | l | l | l |}
\hline
Name & Parameters & Pre-conditions & Post-conditions\\
\hline
		boolean writeTasks 			& String path       			& Requires the path		& Returns a flag to indicate \\
                                                                                    & ArrayList$<$Task$>$ tasks             & and list of tasks 		 &  if save was succesful \\
\hline
		String serialize 			& ArrayList$<$Task$>$ tasks		& Requires list of tasks	& Returns serialized tasks
\\
\hline
\end{tabular}
\\

Attributes: \emph{none}

Purpose: Its purpose is write all the taks to the specified XML file path.
\\
\\

%%%%%%%%%%%%%%%%%%%%%%%%%%%
\emph{XPathHelper}\\
Functions:\\
\begin{tabular}{| l | l | l | l |}
\hline
Name & Parameters & Pre-conditions & Post-conditions\\
\hline
		Document getDocument 		& Object xmlFile       			& Requires the an object	& Returns the XML\\
                                                                                    & 				             & than can be the filepath    	& document \\
							& 					& or an InputStream		& \\
\hline
		Document getDocumentContext	& Object xmlFile       			& Requires the an object	& Returns the XML \\
                                                                                    & 				             & than can be the filepath    	& JXPathContext \\
							& 					& or an InputStream		& \\
\hline
		NodeList getNodeList			& Document doc       			& Requires this XML 		 & Returns the list \\
                                                                                    & String query			            & document and a    		& of nodes matching \\
							& 					& query for nodes		& the specified query
\\
\hline
\end{tabular}
\\

Attributes: \emph{none}

Purpose: Its purpose is load and query any XML file in memory.
\\
\\
\subsection{Subsystem $<$Controller$>$}

\subsubsection{Detailed Design Diagram}
\includegraphics{subsystems/diagrams/controller_class_diagram.jpg}

The controller is responsible for loading the data and giving it to the view.

\subsubsection{Units Description}


\emph{TardisController}\\
Functions:\\
\begin{tabular}{| l | l | l | l |}
\hline
Name & Parameters & Pre-conditions & Post-conditions\\
\hline
\multirow{2}{*}{void main} & String[] args & Requires Nothing & Ensures that the data has been loaded\\ 
			 &  & & and passed to the main view
\\
\hline
\end{tabular}\\
\\
Attributes:\\
\begin{tabular}{| l | l |}
\hline
Name & Description\\
\hline
String PEOPLE\_FILE & The assumed name of the file containing the people.\\
\hline
String TASKS\_FILE & The assumed name of the file containing the tasks.\\
\hline
String VIEW\_FILE & The assumed name of the file in which to dump the people.\\
\hline 
\end{tabular}

Purpose: The main program. Loads the people and tasks from the files and passes them to the main view.
\subsection{Subsystem $<$Views$>$}

\subsubsection{Detailed Design Diagram}
\includegraphics{subsystems/diagrams/views_class_diagram.png}

UML class diagram depicting the internal structure of the subsystem,
accompanied by a paragraph of text describing the rationale of this design.

The purpose of the views subsystem is to display the models cotained in the models subsystem. The idea was to seperate the application and presentation logic. This is used to meet the software engineering goal of loose coupling.

\subsubsection{Units Description}
\emph{PeopleUI}\\
Functions:\\
\begin{tabular}{| l | l | l | l |}
\hline
Name & Parameters & Pre-conditions & Post-conditions\\
\hline
\multirow{2}{*}{PeopleUI}{} & TardisShell shell & Requires Nothing & Ensures that\\ 
			        & ArrayList$<$Task$>$ tasks & & layout has been\\ 
                                            & ArrayList$<$Person$>$ peopleArray & &  setup.
\\
\hline
\multirow{2}{*}{void peopleTablePanel} & None & Requires Nothing & Ensures that\\
& & & the table view\\
& & & for people has\\
& & & been created.
\\
\hline
\multirow{2}{*}{void setPeopleInfo} & None & Requires a valid & Ensures that the\\
		 		            &          & list of people.     & people data is in\\
                                                            &         &                             & the table.
\\
\hline
\multirow{2}{*}{void update} & None & Requires a valid & Ensures that the\\
		                 &          & list of people.     & people data in the\\
		                 &          &                            & table is up to date.
\\
\hline
\end{tabular}\\
\\
Attributes:\\
\begin{tabular}{| l | l |}
\hline
Name & Description\\
\hline
TardisShell shell & The manager of the main view.\\
\hline
ArrayList$<$Task$>$ tasks & The tasks that the user created.\\
\hline
ArrayList$<$Person$>$ people & The people that can be assigned tasks.\\
\hline
String[] columnNames & The field names that will be displayed in the table.\\
\hline
Object[][] peopleInfo & The people in a generic format for the table.\\
\hline
JPanel peopleTablePanel & The panel containing the table.\\
\hline
JTable peopleTable & The table containing the people.\\
\hline
DefaultTableModel model & The generic model used by the JTable.\\
\hline
\end{tabular}

Purpose: Display the people as a table.\\
\\
\\
\emph{TaskUI}\\
Functions:\\
\begin{tabular}{| l | l | l | l |}
\hline
Name & Parameters & Pre-conditions & Post-conditions\\
\hline
\multirow{2}{*}{TaskUI}{} & ITardisShell shell & Requires Nothing & Ensures that the\\ 
			        & ArrayList$<$Task$>$ tasks & & task layout has been\\ 
                                            & ArrayList$<$Person$>$ people & & setup.
\\
\hline
\multirow{2}{*}{void actionPerformed} & ActionEvent e & Requires that e            & Ensures that\\
                                                                &                        & is a valid action event. & the button press\\
                                                                &                        &                                      & was interpreted.
\\
\hline
\multirow{2}{*}{void delete} & int row         & Requires that row points & Ensures that the task \\
		 	           &                     & to an existing row.          & and its subtasks\\
		 	           &                     &                                         & were deleted.\\
 			           &                     &                                         & Ensures that the task\\
                                               &                     &                                         & view was updated.
\\
\hline
\multirow{2}{*}{void setTaskInfo} & None & Requires a valid list & Ensures that the task \\
		 		        &          & of tasks.                   & data is in the table. 
\\
\hline
\multirow{2}{*}{void update} & None & Requires a valid list & Ensures that the task\\
		 	             &          & of tasks.                  & data in the table is\\
			             &          &                                 & up to date.
\\
\hline
\multirow{2}{*}{void taskButtonPanel} & None & Requires Nothing & Ensures that the\\
		 	                            &          &                             & add, edit and delete\\
                                                                &          &                             & buttons have been\\
					    &          &                             & created.
\\
\hline
\multirow{2}{*}{void taskTablePanel} & None & Requires Nothing & Ensures that the\\
		 	                         &           &                             & table view was\\
				             &           &                             & created.
\\
\hline
\end{tabular}

Attributes:\\
\begin{tabular}{| l | l |}
\hline
Name & Description\\
\hline
String[] columnNames & The field names that will be displayed in the table.\\
\hline
DefaultTableModel model & The generic model used by the JTable.\\
\hline
ArrayList$<$Person$>$ people & The people that can be assigned tasks.\\
\hline
ITardisShell shell & The manager of the main view.\\
\hline
JPanel taskButtonPanel & The panel containing the add, edit and delete buttons.\\
\hline
Object[][] taskInfo & The tasks in a generic format for the table.\\
\hline
ArrayList$<$Task$>$ tasks & The tasks that the user created.\\
\hline
JTable taskTable & The table containing the tasks.\\
\hline
JPanel taskTablePanel & The panel containing the tasks table.\\
\hline
\end{tabular}\\
\\
Purpose: Display the tasks as a table.\\
\\
\\
\emph{TaskEditor}\\
Functions:\\
\begin{tabular}{| l | l | l | l |}
\hline
Name & Parameters & Pre-conditions & Post-conditions\\
\hline
\multirow{2}{*}{TaskEditor} & ITardisShell shell                         & Requires Nothing & Ensures that the task\\ 
			          & ArrayList$<$Task$>$ tasks       &                             & editor has been created.\\ 
                                              & ArrayList$<$Person$>$ people &                             & 
\\
\hline
\multirow{2}{*}{TaskEditor} & ITardisShell shell                         & Requires Nothing & Ensures that the task\\ 
			          & ArrayList$<$Task$>$ tasks       &                             & editor has been created\\ 
                                              & ArrayList$<$Person$>$ people &                             & and populated with\\
			          & int index                                      &                            & the data of the\\
			          &                                                     &                            & selected element.
\\
\hline
\multirow{2}{*}{void actionPerformed} & ActionEvent e & Requires that e            & Ensures that\\
                                                        &                        & is a valid action event. & the button press\\
                                                        &                        &                                      & was interpreted.
\\
\hline
\multirow{2}{*}{bool isInteger} & String in & Requires Nothing & Ensures that the\\
		 	                &               &                             & input is a valid\\
		 	                &               &                             & integer.
\\
\hline
\multirow{2}{*}{void taskCreator} & long taskId                    & Requires Nothing & Ensures that either\\
		 	             & String title                     &                             & a new task was created\\
                                                 & String shortDescription &                             & or an existing task\\
				 & int duration                   &                             & was modified.\\
				 & String deliverable          &                             &\\
		       		 & Date dueDate               &                             &\\
				 & int personID                  &                             &\\
				 & Task parent                   &                             &
\\
\hline
\multirow{2}{*}{void updateTask} & None & Requires Nothing & Ensures that the task\\
		                               &           &                             & has been modified.
\\
\hline
\multirow{2}{*}{bool validateDate} & None & Requires Nothing & Ensures that the date\\
		                                  &           &                            & is valid. 
\\
\hline
\multirow{2}{*}{bool validateDuration} & None & Requires Nothing & Ensures that the\\
		                                        &           &                            & duration is valid.
\\
\hline
\multirow{2}{*}{bool validateTitle} & None & Requires Nothing & Ensures that the title\\
		                                &           &                             & is valid.
\\
\hline
\end{tabular}

Attributes:\\
\begin{tabular}{| l | l |}
\hline
JButton CANCEL & The button to close the task window without making changes.\\
\hline
JComboBox cPeople & The control to select who the task is assigned to.\\
\hline
JComboBox cSuper & The control to select the super task.\\
\hline
JLabel cSuperL & The label containing the supertask description.\\
\hline
JLabel deliverable & The label containing the deliverable description.\\
\hline
JLabel dueDateD & The label containing the description of the day part of the due date.\\
\hline
JLabel dueDateM & The label containing the description of the month part of the due date.\\
\hline
JLabel dueDateY & The label containing the description of the year part of the due date.\\
\hline
JLabel duration & The label containing the description of the duration.\\
\hline
int index & The position in the list of tasks.\\
\hline
JPanel panel & The panel containing the task controls.\\
\hline
ArrayList$<$Person$>$ people & The people that can be assigned tasks.\\
\hline
JLabel personID & The description of the person id.\\
\hline
ITardisShell shell & The manager of the main view.\\
\hline
JLabel shortDesc & The description of the short description.\\
\hline
\multirow{2}{*}{JButton SUBMIT} & The button to save a new task or\\
				       & the modifications made to an existing task.
\\
\hline
JLabel superID & The label containing the supertask id.\\
\hline
JLabel taskID & The label containing the description of the task id.\\
\hline
ArrayList$<$Task$>$ tasks & The tasks that the user created.\\
\hline
TextField tDay & The day part of the due date.\\
\hline
TextField tDeliverable & The deliverable of the task.\\
\hline
TextField tDesc & The description of the task.\\
\hline
TextField tDuration & The duration of the task.\\
\hline
JLabel tID & The label containing the task id.\\
\hline
long tIDNum & The task id.\\
\hline
JLabel title & The descripton of the title.\\
\hline
TextField tMonth & The month part of the due date.\\
\hline
TextField tTitle & The title of the task.\\
\hline
TextField tYear & The year part of the due date.\\
\hline
\end{tabular}\\
\\
Purpose: Add and edit a task.\\
\\
\\
\emph{ITaskView}\\
Functions:\\
\begin{tabular}{| l | l | l | l |}
\hline
Name & Parameters & Pre-conditions & Post-conditions\\
\hline
\multirow{2}{*}{void view} & String path                                 & Requires that path point & Ensures that the tasks have been\\ 
			         & ArrayList$<$Person$>$ people & to a valid location.          & written to the file pointed to\\ 
                                             & ArrayList$<$Task$>$ tasks       &                                         & by path.
\\
\hline
\end{tabular}\\
\\
Purpose: Describe the interface for dumping tasks to a file.
\\
\\
\emph{TaskView}\\
Functions:\\
\begin{tabular}{| l | l | l | l |}
\hline
Name & Parameters & Pre-conditions & Post-conditions\\
\hline
\multirow{2}{*}{void view} & String path                                 & Requires that path point & Ensures that the tasks have been\\ 
			        & ArrayList$<$Person$>$ people & to a valid location.          & written to the file pointed to\\ 
                                            & ArrayList$<$Task$>$ tasks       &                                         & by path.
\\
\hline
\end{tabular}

Attributes: None\\
Purpose: Dump tasks to a file.\\
\\
\\
\emph{IPeopleView}\\
Functions:\\
\begin{tabular}{| l | l | l | l |}
\hline
Name & Parameters & Pre-conditions & Post-conditions\\
\hline
\multirow{2}{*}{void view} & String path                                 & Requires that path point & Ensures that the people have\\ 
			         & ArrayList$<$Person$>$ people & to a valid location.          & been written to the file pointed to\\ 
                                             & ArrayList$<$Task$>$ tasks       &                                         & by path.
\\
\hline
\end{tabular}\\
\\
Purpose: Describe the interface for dumping people to a file.
\\
\\
\emph{PeopleView}\\
Functions:\\
\begin{tabular}{| l | l | l | l |}
\hline
Name & Parameters & Pre-conditions & Post-conditions\\
\hline
\multirow{2}{*}{void view} & String path                                 & Requires that path point & Ensures that the people have\\ 
			        & ArrayList$<$Person$>$ people & to a valid location.          & been written to the file pointed to\\ 
                                            & ArrayList$<$Task$>$ tasks       &                                         & by path.
\\
\hline
\end{tabular}

Attributes: None\\
Purpose: Dump people to a file.\\
\\
\\
\emph{ITardisShell}\\
Functions:\\
\begin{tabular}{| l | l | l | l |}
\hline
Name & Parameters & Pre-conditions & Post-conditions\\
\hline
void update & None & Requires Nothing & Ensures that the people and task panels are updated.
\\
\hline
\end{tabular}\\
\\
Purpose: Describe the interface for updating the main screen.
\\
\\
\emph{TardisShell}\\
Functions:\\
\begin{tabular}{| l | l | l | l |}
\hline
Name & Parameters & Pre-conditions & Post-conditions\\
\hline
\multirow{2}{*}{void createAndShowGUI} & ArrayList$<$Task$>$ tasks        & Requires Nothing & Ensures that the main\\ 
			                                 & ArrayList$<$Person$>$ people &                             & screen is displayed. 
\\
\hline
\multirow{2}{*}{TardisShell} & ArrayList$<$Task$>$ tasks       & Requires Nothing & Ensures that the main\\ 
			          & ArrayList$<$Person$>$ people &                             & screen is setup. 
\\
\hline
\multirow{2}{*}{void peoplePrinter} & None & Requires Nothing & Ensures that the people\\ 
			                       &          &                             & have been dumped to\\
			                       &          &                             & a file.
\\
\hline
\multirow{2}{*}{void taskPrinter} & None & Requires Nothing & Ensures that the tasks\\ 
                       	                               &          &                             & have been dumped to\\
                                                       &          &                             & a file. 
\\
\hline
\multirow{2}{*}{void update} & None & Requires Nothing & Ensures that the main\\ 
			            &           &                             & screen is updated. 
\\
\hline
\end{tabular}\\
\\
Attributes: \\
\begin{tabular}{| l | l |}
\hline
Name & Description\\
\hline
ArrayList$<$Person$>$ people & The list of people that can be assigned tasks.\\
\hline
PeopleUI peoplePanel & The panel containing the people table.\\
\hline
JTabbedPane tabbedPane & The view selection pane.\\
\hline
TaskUI taskPanel & The panel containing the tasks table.\\
\hline
ArrayList$<$Task$>$ tasks & The list of tasks that can be assigned.\\
\hline
\end{tabular}\\
\\
\\
Purpose: Manages and displays the main screen.\\
\\
\\
\subsection{Subsystem <Util>}

\subsubsection{Detailed Design Diagram}

UML class diagram depicting the internal structure of the subsystem,
accompanied by a paragraph of text describing the rationale of this design.

\subsubsection{Units Description}

List each class in this subsystem and write a short description of its purpose,
as well as notes or reminders useful for the programmers who will implement them.
List all attributes and functions of the class.

\section{Dynamic Design Scenarios}

\subsection{Add Scenario}
\includegraphics{diagrams/add_sequence_diagram.png}


\subsection{Delete Scenario}
\includegraphics{diagrams/delete_sequence_diagram.png}


\end{document}
