\subsection{Subsystem $<$Models$>$}

\subsubsection{Detailed Design Diagram}
\includegraphics{subsystems/diagrams/models_class_diagram}

This package stores all the classes that belong to the modeling layer of the system. Modeling classes are classes that are used to read and write the data from/to the XML files. It contains the Person, Task, IPeopleReader, PeopleReader, ITaskReader, TaskReader, ITaskWriter, TaskWriter, FileHelper and XPathHelper classes.

\subsubsection{Units Description}

\emph{Person}\\
Functions:\\
\begin{tabular}{| l | l | l | l |}
\hline
Name & Parameters & Pre-conditions & Post-conditions\\
\hline
		getAddress		& None       			& None			& Returns the address attribue\\
		getCity			& None				& None       	 	& Returns the city attribue \\
		getCountry		& None				& None       	 	& Returns the country attribue \\
		getFirstName		& None				& None       	 	& Returns the firstName attribue \\
		getLastName		& None				& None       	 	& Returns the lastName attribue \\
		getPersonId		& None				& None       	 	& Returns the personId attribue \\
		getPhoneNumber	& None				& None       	 	& Returns the phoneNumber attribue \\
		getPostalCode		& None				& None       	 	& Returns the postalCode attribue \\
		getProvince		& None				& None       	 	& Returns the province attribue \\
		getTasks		& ArrayList$<$Task$>$ tasks	& Requires the list 	& Returns all the tasks for this person \\
					& 				& of all tasks 		& \\
		setAddress		& String address  		& None			& Sets the address attribue\\
		setCity			& String city			& None       	 	& Sets the city attribue \\
		setCountry		& String country		& None       	 	& Sets the country attribue \\
		setFirstName		& String firstName		& None       	 	& Sets the firstName attribue \\
		setLastName		& String lastName		& None       	 	& Sets the lastName attribue \\
		setPersonId		& String personId		& None       	 	& Sets the personId attribue \\
		setPhoneNumber	& String phoneNumber	& None       	 	& Sets the phoneNumber attribue \\
		setPostalCode		& String postalCode		& None       	 	& Sets the postalCode attribue \\
		setProvince		& String province		& None       	 	& Sets the province attribue
\\
\hline
\end{tabular}
\\

Attributes:\\
\begin{tabular}{| l | l |}
\hline
 Name                                       & Description\\
\hline
String address		 	 & Address of this person\\
\hline
String city		 	 & City of this person\\
\hline
String country		 	 & Country of this person\\
\hline
String firstName		  & First name of this person\\
\hline
String lastName		  & Last name of this person\\
\hline
String personId		  & Id of this person\\
\hline
String phoneNumber		  & Phone number of this person\\
\hline
String postalCode		  & Postal code of this person\\
\hline
String province		  & Province of this person\\
\hline
\end{tabular}\\
\\

Purpose: Its purpose is to hold the information that is read from the XML file for a particular person.
\\
\\

\emph{Task}\\
Functions:\\
\begin{tabular}{| l | l | l | l |}
\hline
Name & Parameters & Pre-conditions & Post-conditions\\
\hline
		Task		& None       			& None			& Default constructor, initializes subtasks \\
		addSubtask		& Task task			& None       	 	& Adds subtask to this task \\
		getDeliverable		& None				& None       	 	& Returns the country attribue \\
		getDueDate		& None				& None       	 	& Returns the firstName attribue \\
		getDuration		& None				& None       	 	& Returns the lastName attribue \\
		getPersonId		& None				& None       	 	& Returns the personId attribue \\
		getShortDescription	& None				& None       	 	& Returns the phoneNumber attribue \\
		getSubtasks		& None				& None       	 	& Returns the postalCode attribue \\
		getSuperTask		& None				& None       	 	& Returns the province attribue \\
		getTaskId		& None		  		& None			& Returns the province attribue \\
		getTitle		& None				& None       	 	& Returns the province attribue \\
		setDeliverable		& String deliverable		& None       	 	& Sets the deliverable attribue \\
		setDueDate		& String dueDate		& None       	 	& Sets the dueDate attribue \\
		setDuration		& String duration		& None       	 	& Sets the duration attribue \\
		setPersonId		& String personId		& None       	 	& Sets the personId attribue \\
		setShortDescription	& String shortDescription	& None       	 	& Sets the shortDescription attribue \\
		setSuperTask		& String superTask		& None       	 	& Sets the superTask attribue \\
		setTaskId		& String taskId	  		& None			& Sets the taskId attribue \\
		setTitle			& String title			& None       	 	& Sets the title attribue
\\
\hline
\end{tabular}
\\

Attributes:\\
\begin{tabular}{| l | l |}
\hline
 Name                                     	 	& Description\\
\hline
String deliverable			& Deliverable of this task\\
\hline
String dueDate			& Due date of this task\\
\hline
String duration		 		& Duration of this task\\
\hline
String personId			& ID of the person this task belongs to\\
\hline
String shortDescription		& Short description of this task\\
\hline
ArrayList$<$Task$>$ subTasks	& Sub-tasks of this task\\
\hline
String superTask			& Super task of this task\\
\hline
\end{tabular}\\
\\

Purpose: Its purpose is to hold the information that is read from the XML file for a particular task and its sub-tasks.
\\
\\

\emph{IPeopleReader}\\
Functions:\\
\begin{tabular}{| l | l | l | l |}
\hline
Name & Parameters & Pre-conditions & Post-conditions\\
\hline
		ArrayList$<$Person$>$loadPeople 	& String path       & Requires the path 	& Returns the list of people\\
                                                                                    &                          & to the people XML file	& 
\\
\hline
\end{tabular}
\\

Purpose: The interface for loading the people XML file.
\\
\\
\emph{PeopleReader}\\
Functions:\\
\begin{tabular}{| l | l | l | l |}
\hline
Name & Parameters & Pre-conditions & Post-conditions\\
\hline
		ArrayList$<$Person$>$loadPeople 	& String path       & Requires the path 	& Returns the list of people\\
                                                                                    &                          & to the people XML file	& 
\\
\hline
\end{tabular}
\\

Attributes: \emph{none}\\

Purpose: Its purpose is load all the people from the specified XML file path.
\\
\\